
Godishnje se u Srbiji proizvede oko 700 000 tona shec1era, od chega su potebe na domac1em trzhishtu oko 200 000 tona godishnje \cite{sugar2019}. Procena vrednosti trzhishta proizvodnje shec1era u Srbiji 2019. godine je iznosila 45.65 miliona americhkih dolara ($USD$). 

U lancu nabavke shec1era, transport predstavlja najskuplju komponentu \cite{anokic2021metaheuristic}. 
Iz tog razloga je neophodno da proces transporta bude izvrshen u shto krac1em roku, po shto manjoj ceni.

U okviru rada je predlozhen informacioni sistem za menad2ment transporta shec1era od fabrike do klijenta. Fokus c1e biti na transportu na teritoriji Republike Srbije.

\subsection{Opis procesa transporta shec1era}

Klijenti iznajmljuju kamion za transport, kao i radnike za prevoz i utovar.

Neophodno je organizovati transport sa shto vec1om ushtedom novca i vremena.

Kamioni za transport se nalze u krugu fabrike, koja ujedno predstavlja pochetnu i krajnju tachku svakog putovanja. Dakle, pri svakom putovanju, jedno vozilo opsluzhuje jednog klijenta i vrac1a se u fabriku. 


\subsection{Uchesnici u sistemu}

Osnovna podela uchesnika u sistemu je na zaposlene i korisnike. Jedino registrovani korisnici imaju pristup sistemu i moguc1nost zahtevanja transporta.

Kategorije zaposlenog osoblja neophodnog za proces transporta:
\begin{itemize}
    \item Administratori - zaduzheni za odrzhavanje rada sistema, kao i za registrovanje novih korisnika (sa kojima je potpisan ugovor o transportu).
    \item Logistichari - zaduzheni za proces obrade zahteva, kao i za komunikaciju sa klijentima.
    \item Vozachi - zaduzheni za prevoz robe.
    \item Magacioneri - zaduzheni za proces utovara.
    \item Najamnici - zaposhljavaju se kao pomoc1ni radnici pri utovaru, ukoliko je to potrebno.
    \item Serviseri - zaduzheni za odrz1avanje vozila.
    
Svaka kategorija mozhe imati jednu ili vishe osoba koje rade isti posao, ukoliko je to potrebno.

\end{itemize}

\subsection{Korish\-c1eni dijagrami i alati}
Tokom izrade rada, korish\-c1eni su dijagrami:
\begin{itemize}
    \item Dijagrami sluchajeva upotrebe.
    
\end{itemize}
 
Za izradu dijagrama je korish\-c1eni alat: \fontencoding{T1}\selectfont
Visual Paradigm Community Edition.
\fontencoding{OT2}\selectfont
