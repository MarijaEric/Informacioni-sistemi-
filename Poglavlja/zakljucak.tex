U okviru projekta je razvijen predlog informacionog sistema za menad2ment transporta shec1era na podruchju Republike Srbije. Razvoj ovakovog sistema je bitan, sa obzirom na to da trzhishte shec1era predstavlja bitan deo industrije u Srbiji, dok je sam proces transporta najskuplja i najkompleksnija komponenta lanca nabavke.

Predlozhena je klijent server arhitektura, sa relacionom bazom podataka, koja bi implementirala najbitnije funkcionalnosti sistema vezanih za logiku i planiranje, kao i poslove vezane za administraciju i odrzhavanje koji su neophodni za funkiconisanje sistema.

\subsection{Dalji razvoj sistema}

Zbog kompleksnosti procesa transporta, broja chlanova tima, kao i ogranichenog vremenskog roka za izradu projekta, predlog informacionog sistema sadrzhi osnovne elemente potrebne za funkcionisanje transportnog preduzec1a. 
Predlozi za dalji razvoj sistema:

\begin{itemize}
    \item Sa obzirom na to da je kolichina shec1era koja se proizvede u Srbiji vec1a nego potrebe domac1eg trzhishta, proshirivanje sistema na inostrani transport bi moglo znatno da unapredi kvalitet sistema.
    \item Proshirivanje sa nabavki na veliko na manje nabavke. Ovakva promena bi imala uticaj na veliki broj komponenti i na samu logistiku transporta. Implicitno bi ovakva promena dovela do promene velikog broja komponenti sistema.

\end{itemize}